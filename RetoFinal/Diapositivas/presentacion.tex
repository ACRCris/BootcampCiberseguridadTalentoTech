\documentclass[aspectratio=169,10pt]{beamer}

% Tema moderno
\usetheme{metropolis}
\metroset{block=fill, progressbar=frametitle}

% Colores cybersecurity
\definecolor{cybergreen}{RGB}{0,230,118}
\definecolor{cyberdark}{RGB}{20,20,30}
\definecolor{cyberblue}{RGB}{0,176,255}
\definecolor{cyberpurple}{RGB}{156,39,176}
\definecolor{alertred}{RGB}{244,67,54}
\definecolor{warnorange}{RGB}{255,152,0}

\setbeamercolor{palette primary}{bg=cyberdark,fg=white}
\setbeamercolor{frametitle}{bg=cyberdark,fg=cybergreen}
\setbeamercolor{progress bar}{fg=cybergreen,bg=cyberdark!50}
\setbeamercolor{alerted text}{fg=alertred}
\setbeamercolor{example text}{fg=cyberblue}
\setbeamercolor{block title}{bg=cyberpurple,fg=white}
\setbeamercolor{block body}{bg=cyberdark!5}
\setbeamercolor{normal text}{fg=cyberdark}

% Paquetes
\usepackage[utf8]{inputenc}
\usepackage[spanish]{babel}
\usepackage[T1]{fontenc}
\usepackage{graphicx}
\usepackage{booktabs}
\usepackage{tikz}
\usetikzlibrary{positioning,arrows.meta,babel,shapes.geometric,calc}

% Ruta de imagenes
\graphicspath{{../Documento/imagenes/}}

% Fuentes mas pequenas para que quepa el contenido
\setbeamerfont{frametitle}{size=\large}
\setbeamerfont{itemize/enumerate body}{size=\footnotesize}
\setbeamerfont{itemize/enumerate subbody}{size=\scriptsize}

% Estilo mejorado para las páginas de sección
\setbeamercolor{section page}{bg=cyberdark,fg=white}
\setbeamerfont{section title}{size=\Huge,series=\bfseries}
\AtBeginSection[]{
  \begin{frame}[plain]
    \begin{tikzpicture}[remember picture,overlay]
      \fill[cyberdark] (current page.south west) rectangle (current page.north east);
      \node[anchor=center,text=cybergreen,font=\Huge\bfseries] at ([yshift=0.5cm]current page.center) {\insertsection};
      \draw[cybergreen,line width=2pt] ([yshift=-0.3cm,xshift=-3cm]current page.center) -- ([yshift=-0.3cm,xshift=3cm]current page.center);
    \end{tikzpicture}
  \end{frame}
}

% Informacion del documento
\title{Fortaleciendo la Red de Empresa XYZ}
\subtitle{Reto Practico Final -- Ciberseguridad}
\author{Cristian Camilo Perez Puentes}
\institute{Bootcamp de Ciberseguridad 2025}
\date{\today}

\begin{document}

% =============================================
% PORTADA
% =============================================
{
\setbeamercolor{background canvas}{bg=cyberdark}
\begin{frame}[plain]
    \begin{tikzpicture}[remember picture,overlay]
        \fill[cyberdark] (current page.south west) rectangle (current page.north east);
        \node[anchor=center,text=cybergreen,font=\Huge\bfseries] at ([yshift=1.5cm]current page.center) {RETO FINAL};
        \node[anchor=center,text=white,font=\Large] at ([yshift=0.3cm]current page.center) {Fortaleciendo la Red de Empresa XYZ};
        \node[anchor=center,text=cyberblue,font=\normalsize] at ([yshift=-0.7cm]current page.center) {Ciberseguridad Nivel Explorador};
        \node[anchor=center,text=white!70,font=\small] at ([yshift=-2cm]current page.center) {Cristian Camilo Perez Puentes};
        \node[anchor=center,text=white!50,font=\footnotesize] at ([yshift=-2.5cm]current page.center) {Bootcamp 2025 | \today};
    \end{tikzpicture}
\end{frame}
}

% =============================================
% INDICE
% =============================================
\begin{frame}{Temas}
    \begin{columns}[T]
        \begin{column}{0.5\textwidth}
            \Large
            \begin{itemize}
                \setlength{\itemsep}{18pt}
                \item Análisis Inicial
                \item Configuraciones Aplicadas
                \item Políticas de Seguridad
            \end{itemize}
        \end{column}
        \begin{column}{0.5\textwidth}
            \Large
            \begin{itemize}
                \setlength{\itemsep}{18pt}
                \item Justificación Teórica
                \item Conclusiones
            \end{itemize}
        \end{column}
    \end{columns}
\end{frame}

% =============================================
% SECCION 1: ANALISIS INICIAL
% =============================================
\section{Analisis Inicial}

\begin{frame}{Escenario Inicial}
    \begin{columns}[T]
        \begin{column}{0.55\textwidth}
            \textbf{\textcolor{alertred}{! Situacion Actual:}}
            \begin{itemize}
                \item Red plana sin segmentacion
                \item Sin firewall ni controles de acceso
                \item Puertos abiertos innecesarios
                \item Vulnerable a ataques
            \end{itemize}
            \vspace{0.3cm}
            \textbf{\textcolor{cybergreen}{+ Objetivos:}}
            \begin{itemize}
                \item Evaluar vulnerabilidades
                \item Redisenar la topologia
                \item Implementar controles
                \item Definir politicas
            \end{itemize}
        \end{column}
        \begin{column}{0.42\textwidth}
            \centering
            \includegraphics[width=0.75\textwidth]{RedInicial.png}\\
            {\scriptsize\textcolor{gray}{Red inicial}}
        \end{column}
    \end{columns}
\end{frame}

\begin{frame}{Vulnerabilidades Detectadas}
    \begin{columns}[T]
        \begin{column}{0.5\textwidth}
            \footnotesize
            \begin{tabular}{@{}lcc@{}}
                \toprule
                \textbf{Vulnerabilidad} & \textbf{Prob.} & \textbf{Riesgo} \\ \midrule
                Sin Firewall & Alta & \textcolor{alertred}{\textbf{Critico}} \\
                Puertos Abiertos & Media & \textcolor{warnorange}{\textbf{Alto}} \\
                Sin Segmentacion & Media & \textcolor{warnorange}{\textbf{Alto}} \\
                \bottomrule
            \end{tabular}
            \vspace{0.3cm}
            
            \begin{block}{\footnotesize Impacto en Triada CIA}
                \scriptsize
                \textbf{\textcolor{cyberpurple}{C}} - Acceso no autorizado\\
                \textbf{\textcolor{cyberblue}{I}} - Modificacion de datos\\
                \textbf{\textcolor{cybergreen}{A}} - DoS y malware
            \end{block}
        \end{column}
        \begin{column}{0.48\textwidth}
            \begin{alertblock}{\scriptsize Sin Firewall}
                \scriptsize Trafico sin inspeccion. Exposicion total.
            \end{alertblock}
            \begin{alertblock}{\scriptsize Puertos Abiertos}
                \scriptsize HTTP/HTTPS en PC Admin innecesario.
            \end{alertblock}
            \begin{alertblock}{\scriptsize Red Plana}
                \scriptsize Movimiento lateral libre.
            \end{alertblock}
        \end{column}
    \end{columns}
\end{frame}

% =============================================
% SECCION 2: CONFIGURACIONES
% =============================================
\section{Configuraciones Aplicadas}

\begin{frame}{Nueva Topologia Segmentada}
    \begin{columns}[c]
        \begin{column}{0.42\textwidth}
            \centering
            \includegraphics[width=0.8\textwidth]{RedFinalSegmentada.png}\\
            {\scriptsize\textcolor{gray}{Red con VLANs}}
        \end{column}
        \begin{column}{0.55\textwidth}
            \textbf{\textcolor{cybergreen}{+ Mejoras:}}
            \begin{itemize}
                \item Segmentacion en 2 VLANs
                \item Router como gateway y firewall
                \item ACLs para control de trafico
                \item Port Security en switch
                \item WPA2 en red inalambrica
            \end{itemize}
            \vspace{0.2cm}
            \begin{block}{\footnotesize Router-on-a-Stick}
                \scriptsize Subinterfaces G0/0.10 y G0/0.20 con 802.1Q.
            \end{block}
        \end{column}
    \end{columns}
\end{frame}

\begin{frame}{Arquitectura de Red Segmentada}
    \centering
    \begin{tikzpicture}[
        font=\scriptsize,
        >=latex,
        device/.style={
            draw,
            rounded corners,
            minimum width=2.2cm,
            minimum height=0.6cm,
            align=center,
            fill=white
        },
        vlanbox/.style={
            draw,
            rounded corners,
            inner sep=6pt,
            dashed,
            thick
        },
        netlabel/.style={font=\tiny, fill=white, inner sep=1pt},
        scale=0.7, transform shape
    ]
    
    % --- Nube e infraestructura perimetral ---
    \node[device] (internet) {Internet};
    \node[device, below=1cm of internet] (router) {Router ISP\\Firewall + ACLs};
    \node[device, below=1cm of router] (switch) {Switch L2/L3\\Trunk 802.1Q};
    
    % --- VLAN 20 - Invitados ---
    \node[vlanbox, below=2.2cm of switch, xshift=-3.5cm,
          minimum width=3.2cm, minimum height=2.5cm] (vlan20) {};
    \node[anchor=north, font=\tiny\bfseries, yshift=-2pt]
        at (vlan20.north) {VLAN 20 -- Invitados};
    \node[device] at ([yshift=0.2cm]vlan20.center) (apwifi) {WiFi WPA2};
    \node[device, below=0.5cm of apwifi] (pcinv) {PC Invitado};
    
    % --- VLAN 10 - Administrativa ---
    \node[vlanbox, below=2.2cm of switch, xshift=3.5cm,
          minimum width=3.2cm, minimum height=2.5cm] (vlan10) {};
    \node[anchor=north, font=\tiny\bfseries, yshift=-2pt]
        at (vlan10.north) {VLAN 10 -- Admin};
    \node[device] at ([yshift=0.2cm]vlan10.center) (pcadmin) {PC Admin};
    \node[device, below=0.5cm of pcadmin] (web) {Servidor Web};
    
    % --- Conexiones ---
    \draw[thick,->] (internet.south) -- (router.north);
    \draw[thick,->] (router.south) -- (switch.north);
    \draw[thick,->] (switch.south) -- ++(0,-0.3) -| (vlan10.north)
        node[pos=0.25, above, netlabel]{VLAN 10};
    \draw[thick,->] (switch.south) -- ++(0,-0.3) -| (vlan20.north)
        node[pos=0.25, above, netlabel]{VLAN 20};
    \draw[thick, dashed, ->, cybergreen] (pcadmin.south) -- (web.north);
    \draw[thick, dashed, ->, alertred] ([xshift=-3pt]pcinv.west) -- ++(-0.8,0) |- ([yshift=-3pt]internet.west)
        node[pos=0.75, left, netlabel]{Solo web};
    \end{tikzpicture}
\end{frame}

\begin{frame}{Segmentacion en VLANs}
    \begin{columns}[T]
        \begin{column}{0.48\textwidth}
            \begin{exampleblock}{\small VLAN 10 -- Administrativa}
                \scriptsize
                \begin{itemize}
                    \item PC Admin + Servidor Web
                    \item Red: \texttt{192.168.10.0/24}
                    \item Gateway: \texttt{192.168.10.1}
                \end{itemize}
            \end{exampleblock}
            \vspace{0.2cm}
            \begin{block}{\small VLAN 20 -- Invitados}
                \scriptsize
                \begin{itemize}
                    \item Access Point + PC Invitado
                    \item Red: \texttt{192.168.20.0/24}
                    \item Solo acceso a Internet
                \end{itemize}
            \end{block}
        \end{column}
        \begin{column}{0.48\textwidth}
            \textbf{Switch:}
            \scriptsize
            \begin{itemize}
                \item \texttt{Fa0/1}: Trunk 802.1Q
                \item \texttt{Fa0/2-3}: VLAN 10
                \item \texttt{Fa0/4}: VLAN 20
            \end{itemize}
            \vspace{0.2cm}
            \textbf{ACL en Router:}
            \scriptsize
            \begin{itemize}
                \item \textcolor{alertred}{Deny}: VLAN20 $\rightarrow$ VLAN10
                \item \textcolor{cybergreen}{Permit}: VLAN20 $\rightarrow$ Internet
            \end{itemize}
        \end{column}
    \end{columns}
\end{frame}

\begin{frame}{Mecanismos de Autenticacion}
    \begin{columns}[T]
        \begin{column}{0.48\textwidth}
            \begin{block}{\small WPA2-PSK en WiFi}
                \scriptsize
                \begin{itemize}
                    \item SSID: \texttt{Invitados}
                    \item Cifrado: AES
                    \item Passphrase compleja
                \end{itemize}
            \end{block}
            \vspace{0.1cm}
            \begin{block}{\small Acceso a Dispositivos}
                \scriptsize
                \begin{itemize}
                    \item Usuario local privilegio 15
                    \item SSH para gestion remota
                    \item \texttt{enable secret}
                \end{itemize}
            \end{block}
        \end{column}
        \begin{column}{0.48\textwidth}
            \begin{exampleblock}{\small Port Security}
                \scriptsize
                \begin{itemize}
                    \item Max 2 MACs por puerto
                    \item Aprendizaje \texttt{sticky}
                    \item Violacion: \texttt{restrict}
                \end{itemize}
            \end{exampleblock}
            \vspace{0.1cm}
            \textbf{\textcolor{cybergreen}{Beneficios:}}
            \scriptsize
            \begin{itemize}
                \item Impide dispositivos no autorizados
                \item Protege contra MAC spoofing
            \end{itemize}
        \end{column}
    \end{columns}
\end{frame}

% =============================================
% SECCION 3: POLITICAS
% =============================================
\section{Politicas de Seguridad}

\begin{frame}{Higiene Digital}
    \begin{columns}[T]
        \begin{column}{0.48\textwidth}
            \begin{block}{\small Contrasenas}
                \scriptsize
                \begin{itemize}
                    \item Minimo 10 caracteres
                    \item Combinacion de tipos
                    \item Cambio cada 90 dias
                \end{itemize}
            \end{block}
            \vspace{0.1cm}
            \begin{block}{\small Actualizaciones}
                \scriptsize
                \begin{itemize}
                    \item Firmware de dispositivos
                    \item Software de servidores
                    \item Pruebas previas
                \end{itemize}
            \end{block}
        \end{column}
        \begin{column}{0.48\textwidth}
            \begin{exampleblock}{\small Buenas Practicas}
                \scriptsize
                \begin{itemize}
                    \item Bloqueo de pantalla
                    \item Cuentas individuales
                    \item Cierre de sesiones
                \end{itemize}
            \end{exampleblock}
            \vspace{0.1cm}
            \begin{exampleblock}{\small Respaldos}
                \scriptsize
                \begin{itemize}
                    \item Configuraciones de red
                    \item Datos criticos
                    \item Almacenamiento externo
                \end{itemize}
            \end{exampleblock}
        \end{column}
    \end{columns}
\end{frame}

\begin{frame}{Politica de Seguridad}
    \begin{columns}[T]
        \begin{column}{0.48\textwidth}
            \begin{block}{\small Normas de Uso}
                \scriptsize
                \begin{itemize}
                    \item Solo software autorizado
                    \item Uso laboral de recursos
                    \item Proteccion de info confidencial
                \end{itemize}
            \end{block}
            \vspace{0.1cm}
            \begin{block}{\small Control de Acceso}
                \scriptsize
                \begin{itemize}
                    \item Minimo privilegio
                    \item Cuentas individuales
                    \item Trazabilidad
                \end{itemize}
            \end{block}
        \end{column}
        \begin{column}{0.48\textwidth}
            \begin{alertblock}{\small Gestion de Incidentes}
                \scriptsize
                \begin{itemize}
                    \item Reporte inmediato a TI
                    \item Documentar evento
                    \item Escalamiento
                \end{itemize}
            \end{alertblock}
            \vspace{0.1cm}
            \begin{exampleblock}{\small Responsabilidades}
                \scriptsize
                \begin{itemize}
                    \item Usuarios: cumplir normas
                    \item Admins: implementar controles
                \end{itemize}
            \end{exampleblock}
        \end{column}
    \end{columns}
\end{frame}

% =============================================
% SECCION 4: JUSTIFICACION
% =============================================
\section{Justificacion Teorica}

\begin{frame}{Triada CIA}
    \centering
    \begin{tikzpicture}[
        pillar/.style={
            draw,
            rounded corners,
            minimum width=3.3cm,
            minimum height=1.6cm,
            align=center,
            font=\scriptsize,
            text width=3.1cm
        }
    ]
        \node[pillar, fill=cyberpurple!20, draw=cyberpurple] (conf) at (0,0) {
            \textbf{\normalsize\textcolor{cyberpurple}{Confidencialidad}}\\[2pt]
            Segmentacion VLAN\\
            ACLs restrictivas\\
            WPA2-PSK
        };
        \node[pillar, fill=cyberblue!20, draw=cyberblue] (int) at (4,0) {
            \textbf{\normalsize\textcolor{cyberblue}{Integridad}}\\[2pt]
            Autenticacion local\\
            SSH para gestion\\
            MAC sticky
        };
        \node[pillar, fill=cybergreen!20, draw=cybergreen!70!black] (disp) at (8,0) {
            \textbf{\normalsize\textcolor{cybergreen!70!black}{Disponibilidad}}\\[2pt]
            Aislamiento VLANs\\
            Respaldos periodicos\\
            Plan de incidentes
        };
    \end{tikzpicture}
\end{frame}

\begin{frame}{Defensa en Profundidad}
    \centering
    \begin{tikzpicture}[
        layer/.style={
            draw,
            rounded corners,
            minimum height=0.65cm,
            align=center,
            font=\scriptsize
        }
    ]
        \node[layer, fill=alertred!30, minimum width=10cm] (l1) at (0,2) {\textbf{Perimetro:} Router ISP + ACLs + NAT};
        \node[layer, fill=warnorange!30, minimum width=8cm] (l2) at (0,1.2) {\textbf{Red:} Segmentacion VLAN + 802.1Q};
        \node[layer, fill=cyberblue!30, minimum width=6cm] (l3) at (0,0.4) {\textbf{Acceso:} Port Security + WPA2};
        \node[layer, fill=cybergreen!30, minimum width=4cm] (l4) at (0,-0.4) {\textbf{Datos:} Politicas + Respaldos};
        
        \node[font=\scriptsize, align=left, text width=4cm] at (6.5,0.8) {
            \textbf{Principio:}\\
            Multiples capas protegen los activos.\\[3pt]
            Si una falla, las demas contienen.
        };
    \end{tikzpicture}
\end{frame}

\begin{frame}{Minimo Privilegio}
    \begin{columns}[T]
        \begin{column}{0.48\textwidth}
            \begin{exampleblock}{\small En la Red}
                \scriptsize
                \begin{itemize}
                    \item Invitados: solo Internet
                    \item VLAN 10 aislada de VLAN 20
                    \item Servidor: solo HTTPS
                    \item Puertos: MACs conocidas
                \end{itemize}
            \end{exampleblock}
        \end{column}
        \begin{column}{0.48\textwidth}
            \begin{block}{\small En Usuarios}
                \scriptsize
                \begin{itemize}
                    \item Privilegios especificos
                    \item Solo admins gestionan red
                    \item Acceso segun funcion
                    \item Cuentas individuales
                \end{itemize}
            \end{block}
        \end{column}
    \end{columns}
    
    \vspace{0.4cm}
    \centering
    \begin{tikzpicture}
        \node[draw=cybergreen, rounded corners, fill=cybergreen!10, text width=9cm, align=center, font=\small] {
            \textit{``Solo los privilegios minimos necesarios para la funcion.''}
        };
    \end{tikzpicture}
\end{frame}

% =============================================
% SECCION 5: CONCLUSIONES
% =============================================
\section{Conclusiones}

\begin{frame}{Resumen de Mitigaciones}
    \centering
    \footnotesize
    \begin{tabular}{@{}lll@{}}
        \toprule
        \textbf{Vulnerabilidad} & \textbf{Mitigacion} & \textbf{Capa} \\ \midrule
        Sin Firewall & ACLs en RouterISP & Perimetral \\
        Puertos Abiertos & Solo HTTPS en servidor & Config \\
        Sin Segmentacion & VLAN 10 + VLAN 20 & Red \\
        Acceso libre & Port Security + WPA2 & Acceso \\
        Sin politicas & Documento normativo & Organizacional \\
        \bottomrule
    \end{tabular}
    
    \vspace{0.5cm}
    \begin{tikzpicture}
        \node[draw=cybergreen, rounded corners, fill=cybergreen!10, text width=10cm, align=center, font=\small] {
            \textbf{Resultado:} Red segmentada con controles multinivel, alineada con CIA.
        };
    \end{tikzpicture}
\end{frame}

\begin{frame}{Recomendaciones Futuras}
    \begin{columns}[T]
        \begin{column}{0.48\textwidth}
            \begin{block}{\small Corto Plazo}
                \scriptsize
                \begin{itemize}
                    \item Implementar 802.1X (RADIUS)
                    \item Configurar IDS/IPS
                    \item Centralizar logs
                    \item Automatizar respaldos
                \end{itemize}
            \end{block}
        \end{column}
        \begin{column}{0.48\textwidth}
            \begin{exampleblock}{\small Mediano Plazo}
                \scriptsize
                \begin{itemize}
                    \item Firewall dedicado (NGFW)
                    \item VPN para acceso remoto
                    \item SIEM centralizado
                    \item Auditorias periodicas
                \end{itemize}
            \end{exampleblock}
        \end{column}
    \end{columns}
    
    \vspace{0.5cm}
    \centering
    \begin{tikzpicture}
        \node[draw=cyberblue, rounded corners, fill=cyberblue!10, text width=8cm, align=center, font=\small] {
            \textbf{La seguridad es un proceso continuo.}
        };
    \end{tikzpicture}
\end{frame}

% =============================================
% CIERRE
% =============================================


\end{document}
