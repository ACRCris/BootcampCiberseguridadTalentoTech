% Ejemplo de sección modular para vulnerabilidades
% Puedes incluir este archivo en el documento principal con % Ejemplo de sección modular para vulnerabilidades
% Puedes incluir este archivo en el documento principal con % Ejemplo de sección modular para vulnerabilidades
% Puedes incluir este archivo en el documento principal con % Ejemplo de sección modular para vulnerabilidades
% Puedes incluir este archivo en el documento principal con \input{secciones/vulnerabilidades.tex}

\subsection{Ejemplo de Vulnerabilidad Detallada}

\subsubsection{SQL Injection en Formulario de Login}

\begin{itemize}[leftmargin=2cm]
    \item \textbf{Identificador:} VULN-2025-001
    \item \textbf{Descripción:} El formulario de autenticación no valida correctamente las entradas del usuario, permitiendo la inyección de código SQL malicioso.
    \item \textbf{Severidad:} \textcolor{red}{CRÍTICA}
    \item \textbf{CVSS Score:} 9.8 (CVSS:3.1/AV:N/AC:L/PR:N/UI:N/S:U/C:H/I:H/A:H)
    \item \textbf{Sistema Afectado:} Aplicación web principal - módulo de autenticación
    \item \textbf{Ubicación:} \texttt{/login.php} línea 45
    \item \textbf{Impacto:} 
        \begin{itemize}
            \item Acceso no autorizado a la base de datos
            \item Exposición de credenciales de usuarios
            \item Potencial compromiso total del sistema
        \end{itemize}
    \item \textbf{Prueba de Concepto:}
\begin{lstlisting}[language=SQL]
' OR '1'='1' --
admin' UNION SELECT NULL, username, password FROM users --
\end{lstlisting}
    \item \textbf{Recomendación:} 
        \begin{enumerate}
            \item Implementar prepared statements con parámetros vinculados
            \item Validar y sanitizar todas las entradas del usuario
            \item Aplicar el principio de mínimos privilegios en las consultas
            \item Implementar WAF (Web Application Firewall)
        \end{enumerate}
    \item \textbf{Referencias:}
        \begin{itemize}
            \item OWASP A03:2021 - Injection
            \item CWE-89: SQL Injection
        \end{itemize}
\end{itemize}


\subsection{Ejemplo de Vulnerabilidad Detallada}

\subsubsection{SQL Injection en Formulario de Login}

\begin{itemize}[leftmargin=2cm]
    \item \textbf{Identificador:} VULN-2025-001
    \item \textbf{Descripción:} El formulario de autenticación no valida correctamente las entradas del usuario, permitiendo la inyección de código SQL malicioso.
    \item \textbf{Severidad:} \textcolor{red}{CRÍTICA}
    \item \textbf{CVSS Score:} 9.8 (CVSS:3.1/AV:N/AC:L/PR:N/UI:N/S:U/C:H/I:H/A:H)
    \item \textbf{Sistema Afectado:} Aplicación web principal - módulo de autenticación
    \item \textbf{Ubicación:} \texttt{/login.php} línea 45
    \item \textbf{Impacto:} 
        \begin{itemize}
            \item Acceso no autorizado a la base de datos
            \item Exposición de credenciales de usuarios
            \item Potencial compromiso total del sistema
        \end{itemize}
    \item \textbf{Prueba de Concepto:}
\begin{lstlisting}[language=SQL]
' OR '1'='1' --
admin' UNION SELECT NULL, username, password FROM users --
\end{lstlisting}
    \item \textbf{Recomendación:} 
        \begin{enumerate}
            \item Implementar prepared statements con parámetros vinculados
            \item Validar y sanitizar todas las entradas del usuario
            \item Aplicar el principio de mínimos privilegios en las consultas
            \item Implementar WAF (Web Application Firewall)
        \end{enumerate}
    \item \textbf{Referencias:}
        \begin{itemize}
            \item OWASP A03:2021 - Injection
            \item CWE-89: SQL Injection
        \end{itemize}
\end{itemize}


\subsection{Ejemplo de Vulnerabilidad Detallada}

\subsubsection{SQL Injection en Formulario de Login}

\begin{itemize}[leftmargin=2cm]
    \item \textbf{Identificador:} VULN-2025-001
    \item \textbf{Descripción:} El formulario de autenticación no valida correctamente las entradas del usuario, permitiendo la inyección de código SQL malicioso.
    \item \textbf{Severidad:} \textcolor{red}{CRÍTICA}
    \item \textbf{CVSS Score:} 9.8 (CVSS:3.1/AV:N/AC:L/PR:N/UI:N/S:U/C:H/I:H/A:H)
    \item \textbf{Sistema Afectado:} Aplicación web principal - módulo de autenticación
    \item \textbf{Ubicación:} \texttt{/login.php} línea 45
    \item \textbf{Impacto:} 
        \begin{itemize}
            \item Acceso no autorizado a la base de datos
            \item Exposición de credenciales de usuarios
            \item Potencial compromiso total del sistema
        \end{itemize}
    \item \textbf{Prueba de Concepto:}
\begin{lstlisting}[language=SQL]
' OR '1'='1' --
admin' UNION SELECT NULL, username, password FROM users --
\end{lstlisting}
    \item \textbf{Recomendación:} 
        \begin{enumerate}
            \item Implementar prepared statements con parámetros vinculados
            \item Validar y sanitizar todas las entradas del usuario
            \item Aplicar el principio de mínimos privilegios en las consultas
            \item Implementar WAF (Web Application Firewall)
        \end{enumerate}
    \item \textbf{Referencias:}
        \begin{itemize}
            \item OWASP A03:2021 - Injection
            \item CWE-89: SQL Injection
        \end{itemize}
\end{itemize}


\subsection{Ejemplo de Vulnerabilidad Detallada}

\subsubsection{SQL Injection en Formulario de Login}

\begin{itemize}[leftmargin=2cm]
    \item \textbf{Identificador:} VULN-2025-001
    \item \textbf{Descripción:} El formulario de autenticación no valida correctamente las entradas del usuario, permitiendo la inyección de código SQL malicioso.
    \item \textbf{Severidad:} \textcolor{red}{CRÍTICA}
    \item \textbf{CVSS Score:} 9.8 (CVSS:3.1/AV:N/AC:L/PR:N/UI:N/S:U/C:H/I:H/A:H)
    \item \textbf{Sistema Afectado:} Aplicación web principal - módulo de autenticación
    \item \textbf{Ubicación:} \texttt{/login.php} línea 45
    \item \textbf{Impacto:} 
        \begin{itemize}
            \item Acceso no autorizado a la base de datos
            \item Exposición de credenciales de usuarios
            \item Potencial compromiso total del sistema
        \end{itemize}
    \item \textbf{Prueba de Concepto:}
\begin{lstlisting}[language=SQL]
' OR '1'='1' --
admin' UNION SELECT NULL, username, password FROM users --
\end{lstlisting}
    \item \textbf{Recomendación:} 
        \begin{enumerate}
            \item Implementar prepared statements con parámetros vinculados
            \item Validar y sanitizar todas las entradas del usuario
            \item Aplicar el principio de mínimos privilegios en las consultas
            \item Implementar WAF (Web Application Firewall)
        \end{enumerate}
    \item \textbf{Referencias:}
        \begin{itemize}
            \item OWASP A03:2021 - Injection
            \item CWE-89: SQL Injection
        \end{itemize}
\end{itemize}
