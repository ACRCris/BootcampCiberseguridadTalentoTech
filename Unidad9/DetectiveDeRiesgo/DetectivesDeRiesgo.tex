% Documento: Tabla para la actividad "Detectives de Riesgo"
\documentclass[12pt,a4paper]{article}
\usepackage[utf8]{inputenc}
\usepackage[T1]{fontenc}
\usepackage[spanish]{babel}
\usepackage{geometry}
\usepackage{tabularx}
\usepackage{array}
\usepackage{booktabs}
\usepackage{longtable}
\usepackage{makecell}
\usepackage{pdflscape}
\geometry{margin=1cm}

\begin{document}

\begin{landscape}

\begin{center}
	{\LARGE \textbf{Detectives de Riesgo}}\\[6pt]
	{\small Tabla: Matriz rápida de riesgos (sin soluciones)}\\[8pt]
	{\footnotesize \textbf{Integrantes:} Cristian Camilo Pérez, Karen Duarte, Eric Santiago Baquero}
\end{center}

\vspace{8pt}

% Tabla de la actividad: 5 filas vacías para que los estudiantes completen
\renewcommand{\arraystretch}{1.35}
\begin{longtable}{@{}p{0.20\textwidth}p{0.15\textwidth}p{0.15\textwidth}p{0.15\textwidth}p{0.25\textwidth}@{}}
\toprule
\makecell{\textbf{Riesgo} \\ \textbf{identificado}} & \makecell{\textbf{Probabilidad}} & \makecell{\textbf{Impacto}} & \makecell{\textbf{Nivel de} \\ \textbf{riesgo}} & \makecell{\textbf{Medida} \\ \textbf{preventiva}}\\
\midrule
\endfirsthead
\toprule
\makecell{\textbf{Riesgo} \\ \textbf{identificado}} & \makecell{\textbf{Probabilidad}} & \makecell{\textbf{Impacto}} & \makecell{\textbf{Nivel de} \\ \textbf{riesgo}} & \makecell{\textbf{Medida} \\ \textbf{preventiva}}\\
\midrule
\endhead
% Filas vacías para completar
Phishing & Alta & Alta & Crítico & Capacitar personal para no accedan o descarguen archivos de orígenes desconocidos y aplicar MFA \\
\addlinespace
Malware & Medio & Alta & Alto & No introducir USB a los equipos corporativos, restringir la navegación \\
\addlinespace
Mishing & Alto & Medio & Alto & Restringir el uso de celulares personales para usos corporativo, con el uso de herramientas corporativas que impidan el acceso a recursos empresariales y solo de dispositivos autorizados \\
\addlinespace
Contraseña débil en servidor y correo electrónico & Bajo & Alto & Medio & Mejorar y reforzar las contraseñas de los usuarios y servidores, tener políticas de contraseñas fuertes \\
\addlinespace
Pérdida / Robo de información y de celulares & Media & Alta & Alto & Realizar respaldos semanales del servidor \\
\addlinespace
\bottomrule
\end{longtable}

\vspace{10pt}

% Análisis del riesgo más importante
\noindent\textbf{Análisis del riesgo más importante:}\\[6pt]
\noindent El riesgo más importante es el Phishing ya que cuenta con un nivel de riesgo Crítico y probabilidad es muy alta ya que el gerente ha observado ataques de phishing\\[6pt]
\noindent \textbf{Acción para reducirlo:} La acción para reducirlo podría ser capacitar personal para no accedan o descarguen archivos de orígenes desconocidos y aplicar MFA

\end{landscape}

\end{document}

